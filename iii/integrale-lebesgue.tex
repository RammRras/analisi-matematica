\chapter{Misura e integrale di Lebesgue}
Avevamo visto nel capitolo \ref{ch:integrale-riemann} la teoria dell'integrazione sviluppata da Riemann.
Essa è utile nel calcolo integrale di molte funzioni comuni, ma risulta ancora limitata: la classe delle funzioni integrabili secondo Riemann è ampia, ma non comprende ad esempio i limiti delle successioni, difficili da analizzare in tale contesto.
L'integrale di Lebesgue si comporta meglio su questo tipo di funzioni, permettendo condizioni meno restrittive, e per questo risulta molto più adatta in questo campo.
Le funzioni integrabili secondo Lebesgue comprendono anche casi una volta considerati ``patologici'' come la funzione di Dirichlet: essa ha un integrale di Lebesgue, che è anche piuttosto semplice da calcolare, ma non di Riemann.

La costruzione dell'integrale di Lebesgue si fonda su una nuova teoria della misura, che permette di definire il concetto di insieme misurabile e funzione misurabile: si vedrà che trovare insiemi (e funzioni) non misurabili è molto difficile, ma comunque sarà essenziale distinguere tra le due classi.
