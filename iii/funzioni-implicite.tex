\chapter{Funzioni implicite}
\begin{definizione}
	Una curva è una mappa $\phi\colon I\to\R^2$ che associa a $t$ la coppia ordinata $\big(x(t),y(t)\big)$. Essa si dice regolare se è di classe $\cont{1}$ e se $\phi'(t)=\big(x'(t),y'(t)\big)\neq(0,0)$ $\forall t\in I$.
\end{definizione}
Una funzione $f\colon[a,b]\to\R$ di classe $\cont{1}$ può essere considerata come una curva in $\R^2$ parametrizzata come $\big(t,f(t)\big)$. Essa è sempre regolare, qualunque sia la sua forma analitica, poiché la sua derivata è $\big(1,f'(t)\big)$ che non può ovviamente mai essere uguale a $(0,0)$.
\begin{osservazione}
	Una curva semplice e regolare può essere localmente rappresentata come grafico di una funzione.
\end{osservazione}
\begin{proof}
	Sia $\phi[a,b]\to\R^2$ regolare, cioè sia $\big(x'(t),y'(t)\big)\neq(0,0)$: ciò signfica che $x'(t)$ o $y'(t)$ non è nulla.
	Sia $t_0$ un punto di $[a,b]$ e poniamo per ipotesi $x'(t_0)>0$: per la continuità di $\phi$ deve esistere un intorno $U(t_0)\subset[a,b]$ in cui ancora $x'(t)>0$ $\forall t\in U(t_0)$.
	Definiamo allora $\psi(t)$ la curva $\phi(t)$ ristretta a $U(t_0)$. Poiché in $U(t_0)$ la $x'(t)$ è sempre positiva, significa che $x(t)$ è monotona crescente, quindi invertibile nell'intorno. Chiamiamo l'inversa $t(x)$: allora possiamo rappresentare $\psi(t)$ come la funzione $y=f\big(t(x)\big)=f(t)$.
	\end{proof}
Spesso però una curva non è data in rappresentazione parametrica, ma tramite un'equazione cartesiana che lega le coordinate nella formula $F(x,y)=0$. Ad esempio, la circonferenza unitaria è data dalla funzione $F(x,y)=x^2+y^2-1$: i punti soluzioni dell'equazione $F(x,y)=0$ sono i punti della circonferenza.
Questo esempio è un caso piuttosto facile, ma come facciamo a capire cosa rappresenta l'equazione $x^2(4x+1)-y^2=0$? Non sappiamo, così a priori, se è nemmeno una curva. Il seguente teorema, che è il teorema delle funzioni implicite, fornisce delle condizioni sufficienti a rappresentare certe curve ``nascoste'' in equazioni come queste.
\begin{teorema}[di Dini] \label{t:dini-R2}
	Sia $A\subset\R^2$ un aperto e $F\colon A\to\R$ una funzione di classe $\cont{1}(A)$, e sia $(x_0,y_0)\in A$ uno zero di questa funzione tale che $\drp{F}{y}(x_0,y_0)\neq 0$. Allora esistono degli intorni $U(x_0)$ e $V(y_0)$ per i quali $\forall x\in U$ $\exists! y=f(x)\in V$ tale che $F(x,y)=0$. Inoltre, la mappa $x\mapsto f(x)$ è di classe $\cont{1}$ e vale la relazione
	\begin{equation}
		f'(x)=-\dfrac{\drp{F}{x}\big(x,f(x)\big)}{\drp{F}{y}\big(x,f(x)\big)}.
		\label{eq:dini-R2}
	\end{equation}
\end{teorema}
Localmente nell'intorno dato, quindi, si può rappresentare la curva come funzione $y(x)$ come nell'osservazione precedente.
\begin{proof}
	Supponiamo $\drp{F}{y}(x_0,y_0)>0$: per la continuità di $F$ deve esistere un rettangolo (arbitrariamente piccolo) $R=W\times V$ che contiene $(x_0,y_0)$ e in cui $\drp{F}{y}(x,y)>0$ per ogni $(x,y)\in R$.
	Sia $V=[y_0-h,y_0+h]$, con $h>0$. Fissato un punto qualunque $x$ in $W$, possiamo affermare che la mappa $y\mapsto F(x,y)$ è crescente in $V$ perché la derivata rispetto a $y$ è ivi sempre positiva.
	Di conseguenza abbiamo che $F(x_0,y_0+h)>0$ e $F(x_0,y_0-h)<0$; ma sempre perché $F$ è continua possiamo individuare un intorno $U\subset W$ di $x_0$ tale che $\forall x\in U$ si ha $F(x,y_0+h)>0$ e $F(x,y_0-h)<0$, dunque per queste due ultime disuguaglianze e per il fatto che $F$ è crescente lungo $y$ deve necessariamente esistere un unico zero di $F$ in ogni segmento $x\times V$, con sempre $x\in U$.
	L'insieme di questi zeri forma il grafico della funzione $f(x)=y(x)$ cercata.

	Siano ora $x,x_1\in U$ con $f(x),f(x_1)\in V$; i punti $\big(x,f(x)\big)$ e $\big(x_1,f(x_1)\big)$ sono quindi in $R$. Poiché $F\in\cont{1}$, per il teorema di Lagrange esiste un punto $(a,b)$, appartenente al segmento con estremi questi due punti, in cui
	\begin{equation}
		F\big(x,f(x)\big)-F\big(x_1,f(x_1)\big)=\drp{F}{x}(a,b)(x-x_1)+\drp{F}{y}(a,b)\big(f(x)-f(x_1)\big).
	\end{equation}
	Ma i due punti in cui è valutata $F$ sono dei suoi zeri, e d'altra parte abbiamo $\drp{F}{y}(x,y)>0$ in pgni punto di $R$, quindi anche in questi due punti. Allora
	\begin{gather*} \label{eq:dini-dim1}
		\drp{F}{x}(a,b)(x-x_1)=-\drp{F}{y}(a,b)\big(f(x)-f(x_1)\big)\\
		\frac{\big(f(x)-f(x_1)\big)}{x-x_1}=-\dfrac{\drp{F}{x}(a,b)}{\drp{F}{y}(a,b)}
	\end{gather*}
	Inoltre, le derivate parziali di $F$ sono continue (e lo sono perciò anche i loro moduli) quindi per il teorema \ref{t:weierstrass} di Weierstrass entrambe ammettono massimo e minimo in $R$, che è chiuso e limitato, cioè compatto. Dunque possiamo affermare che
	\begin{equation}
		\abs{f(x)-f(x_1)}\leq\frac{\max_{(x,y)\in R}\abs{\drp{F}{x}(x,y)}}{\min_{(x,y)\in R}\abs{\drp{F}{y}(x,y)}}(x-x_1)
	\end{equation}
	e questo mostra che $f$ è lipschitziana, quindi (uniformemente) continua.
	Passando al limite per $x_1\to x$ nella \eqref{eq:dini-dim1} inoltre otteniamo, dato che $f(x_1)\to f(x)$ per continuità e che $(a,b)\to\big(x,f(x)\big)$, l'equazione \eqref{eq:dini-R2} della tesi.
\end{proof}
Passiamo ora ad una versione di questo teorema in tre dimensioni, relativamente simile al precedente, quindi ad una generalizzazione in un numero arbitrario di dimensioni.
\begin{teorema} \label{t:dini-R3}
	Sia $A\subset\R^3$ un aperto e $F\colon A\to\R$ una funzione di classe $\cont{1}(A)$, e sia $(x_0,y_0,z_0)\in A$ uno zero di questa funzione tale che $\drp{F}{z}(x_0,y_0,z_0)\neq 0$. Allora esiste un intorno $U(x_0,y_0,z_0)$ nel quale i punti per cui $F(x,y,z)=0$ si possono scrivere come funzione $z=f(x,y)$. Inoltre, la mappa $x\mapsto f(x)$ è di classe $\cont{1}$ e vale la relazione
	\begin{equation}
		\drp{f}{x}(x,y)=-\dfrac{\drp{F}{x}\big(x,y,f(x,y)\big)}{\drp{F}{z}\big(x,y,f(x,y)\big)},
	\label{eq:dini-R3}
\end{equation}
	e analogamente per la derivata nella variabile $y$.
\end{teorema}
\begin{teorema} \label{t:dini-Rn}
	Sia $A\subset\R^n\times\R^m$ un aperto, e $(\vec x,\vec y)=(x_1,\dots,x_m,y_1,\dots,y_n)$ un suo punto. Sia $\vec F\colon A\to\R^n$ una funzione di classe $\cont{1}(A)$. Si definiscono le matrici $F_x$ e $F_y$, rispettivamente di dimensioni $n\times m$ e $n\times n$, come
	\[
	F_x=
		\begin{bmatrix}
		\drp{F_1}{x_1}	&\cdots	&\drp{F_1}{x_m}\\
		\vdots			&\ddots	&\vdots\\
		\drp{F_n}{x_1}	&\cdots	&\drp{F_n}{x_m}
		\end{bmatrix}
	\qqq
	F_y=
		\begin{bmatrix}
		\drp{F_1}{y_1}	&\cdots	&\drp{F_1}{y_n}\\
		\vdots			&\ddots	&\vdots\\
		\drp{F_n}{y_1}	&\cdots	&\drp{F_n}{y_n}
		\end{bmatrix}.
	\]
	Sia infine $(\vec x_0,\vec y_0)$ un punto di $A$ tale che $\vec F(\vec x_0,\vec y_0)=\vec 0$ e che $\det F_y(\vec x_0,\vec y_0)\neq 0$. Allora esistono degli intorni $U\in\R^m$ di $\vec x_0$ e $V\in\R^n$ di $\vec y_0$ per i quali $\forall\vec x\in U$ $\exists! \vec y=\vec f(\vec x)\in V$ tale che $\vec F\big(\vec x,\vec f(\vec x)\big)=\vec 0$. Inoltre la funzione $\vec f\colon U\to\R^n$ è di classe $\cont{1}$ e vale la relazione
	\begin{equation} \label{eq:dini-Rn}
		\jac\vec f(\vec x)=-\Big[F_y\big(\vec x,\vec f(\vec x)\big)\Big]^{-1}F_x\big(\vec x,\vec f(\vec x)\big).
	\end{equation}
\end{teorema}
\begin{corollario}
	Se la funzione $F$ è di classe $\cont{k}$ nell'insieme $A$, allora la funzione implicita $f$ del teorema di Dini eredita la regolarità ed è anch'essa di classe $\cont{k}$ nell'intorno $U$ di $x_0$.
\end{corollario}
\begin{proof}
	Semplicemente, poiché la derivata parziale di $F$ in $y$ non è mai nulla, possiamo continuare a derivare la $f'(x)$ per ottenere quelle di grado superiore: al denominatore troveremo sempre una potenza di questa derivata. Gli altri termini saranno le varie derivate parziali, dei vari ordini fino a quello a cui si vuole derivare $f$, che certamente esistono dato che $F\in\cont{k}$, quindi tutte le sue derivate parziali fino al $k$-esimo ordine esistono.
\end{proof}
Se abbiamo $F\in\cont{2}$ che rispetta le ipotesi del teorema di Dini, la derivata seconda di $f$ è
\begin{equation}
	f''(x)=-\frac{\partial_{xx}F+\partial_{xy}Ff'}{\partial_yF+\frac1{(\partial_yF)^2}\partial_xF(\partial_{yx}F+\partial_{yy}Ff')}
\end{equation}
valutata in $\big(x,f(x)\big)$. Sostituendo a $f(x)$ l'espressione della \eqref{eq:dini-R2} si semplifica nella
\begin{equation} 
	f''(x)=-\frac{\partial_{xx}F(\partial_yF)^2-2\partial_xF\partial_{xy}F\partial_yF+\partial_{yy}F(\partial_xF)^2}{(\partial_yF)^3},
	\label{eq:dini-derivata-seconda}
\end{equation}
sfruttando il fatto che $\partial_{xy}F=\partial_{yx}F$ per il teorema di Schwartz, dato che la funzione è di classe $\cont{2}$ le derivate miste coincidono.
Possiamo dunque verificare, se $F(x_0,y_0)$ è nulla, se abbiamo un punto di massimo o minimo (locale) per la funzione implicita. Sappiamo già che la $f$ deve essere derivabile, dato che come conseguenza del teorema di Dini è di classe $\cont{1}$. Allora per il teorema di Fermat in ogni punto estremante la derivata prima deve essere nulla. Se $f'(x_0)=0$, dalla \eqref{eq:dini-R2} segue che $\partial_xF(x_0,y_0)=0$, quindi sostituendolo nella \eqref{eq:dini-derivata-seconda} otteniamo
\begin{equation} 
	f''(x_0)=-\frac{\partial_{xx}F(x_0,y_0)}{\partial_yF(x_0,y_0)},
	\label{eq:derivata-seconda-semplificata-dini}
\end{equation}
da cui segue immediatamente che se è positiva si ha un minimo, se è negativa un massimo. Il punto $x_0$ allora è di minimo locale se $\partial_{xx}F(x_0,y_0)/\partial_yF(x_0,y_0)<0$, un massimo locale se $F(x_0,y_0)/\partial_yF(x_0,y_0)>0$.
La condizione è ovviamente soltanto sufficiente.

\section{Diffeomorfismi}
\begin{definizione} \label{d:diffeomorfismo-locale}
	Siano $A,B\subset\R^n$ insiemi aperti, $\vec x_0$ un punto di $A$ e $\vec f\colon A\to B$ una funzione di classe $\cont{1}(A)$. La funzione $\vec f$ si dice \emph{diffeomorfismo locale} in $\vec x_0$ se esiste un opportuno intorno $U$ di tale punto per il quale la $\vec f$ ristretta a $U$ e alla sua immagine, ossia $\vec f\colon U\to \vec f(U)$, è un diffeomorfismo.
\end{definizione}
Il seguente teorema mostra una condizione sufficiente a determinare se una funzione è un diffeomorfismo locale.
\begin{teorema} \label{t:diffeomorfismo-locale}
	Siano $A,B$ due insiemi aperti di $\R^n$, $\vec x_0$ un punto di $A$, e $\vec f$ una funzione da $A$ a $B$. La $\vec f$ è un diffeomorfismo locale in $\vec x_0$ se la matrice jacobiana in tale punto non è singolare, cioè se
	\begin{equation} 
		\det\jac\vec f(\vec x_0)\neq 0.
		\label{eq:condiz-nec-diffeomorfismo}
	\end{equation}
\end{teorema}
\begin{proof}
	Supponiamo che $\vec f$ non sia iniettiva in qualsiasi intorno di $\vec x_0$. Allora si può scegliere un intorno in cui due punti in esso hanno la stessa immagine: siano essi $\vec x_n$ e $\tilde{\vec x_n}$, allora $\vec f(\vec x_n)=\vec f(\tilde{\vec x_n})$.
	Questo vale per ogni intorno di $\vec x_0$, perciò posiamo immaginare che $\vec x_n$ e $\tilde{\vec x_n}$ siano due successioni convergenti a $\vec x_0$.
	Per il teorema di Lagrange, per ogni $i$-esima componente $\vec f$ si ha che esiste un punto $\vxi_i$ appartenente al segmento $[\vec x_n, \tilde{\vec x_n}]$ tale per cui abbiamo che
	\begin{equation}
		f_i(\vec x_n)-f_i(\tilde{\vec x_n})=\scalar{\grad f_i(\vxi_i)}{\vec x_n-\tilde{\vec x_n}}
		\label{eq:lagrange-in-diffeomorfismo-locale}
	\end{equation}
	Per quanto detto prima, però, questo è nullo perché $\vec f(\vec x_n)=\vec f(\tilde{\vec x_n})$.
	Allora possiamo dividere per $\norm{\vec x_n-\tilde{\vec x_n}}$ ottenendo che
	\begin{equation}
		\scalar{\grad f_i(\vxi_i)}{ \frac{\vec x_n-\tilde{\vec x_n}} {\norm{\vec x_n-\tilde{\vec x_n}}} }=0.
		\label{eq:vettori-ortogonali-diffeomorfismo}
	\end{equation}
	Chiamiamo $\vec v_n$ il vettore $\vec x_n-\tilde{\vec x_n}$ normalizzato: esso è un versore, che appartiene dunque alla sfera $n$ dimensionale $S^{n+1}\subset\R^n$.
	Essa è un insieme compatto dunque $\vec v_n$, che è una successione in esso contenuta, deve convergere ad un elemento di $S^{n-1}$.
	Allora $\scalar{\grad f(\vxi_i)}{\vec v}=0$ per ogni $i$, vale a dire $\jac\vec f(\vec x_0)\vec v=\vec 0$.
	Questo però è assurdo, poiché $\vec v$ non è nullo e $\det\jac\vec f(\vec x_0)\neq 0$.
	Allora $\vec f$ deve essere iniettiva\footnote{Poiché $\det\jac\vec f(\vec x_0)\neq 0$, essa individua un operatore lineare iniettivo, ossia $\ker\jac\vec f(\vec x_0)=\{\vec 0\}$, quindi l'unico vettore che può dare il vettore nullo è solo il vettore nullo stesso (e $\vec v$ non lo è!).}.

	Poniamo adesso $\vec F(\vec x,\vec y)=\vec y-\vec f(\vec x)$, dove $\vec F$ è una funzione da $W\times\R^n$ a $\R^n$, dove $W$ è un intorno di $\vec x_0$, e sia inoltre $\vec y_0=\vec f(\vec x_0)$.
	Sicuramente $\vec F\in\cont{1}$, e $\vec F(\vec x_0,\vec y_0)=\vec 0$.
	Risulta che $\det\jac\vec F(\vec x_0,\vec y_0)=-\det\jac\vec f(\vec x_0)\neq 0$.
	Il teorema di Dini afferma che esiste un intorno $V$ di $\vec y_0$ tale che esiste una funzione $\vphi\colon V\to W$ di classe $\cont{1}(V)$ per cui vale $\vec F(\vphi(\vec y),\vec y)=\vec 0$ per ogni $\vec v\in V$.
	Allora $\vec y=\vec f\big(\vphi(\vec y)\big)$, cioè $\vec y=\vec f\circ \vphi(\vec y)$ ossia $\vphi$ è l'inversa di $\vec f$, ammesso che una sua inversa esista: dimostriamolo.
	Chiamiamo $U\defeq W\cap\vec f^{-1}(V)$. Per tale insieme vale che $\vec f(\vec x)=\vec y\in V$, quindi
	\begin{equation*}
		\vec f\big(\vphi(\vec y)\big)=\vec f\Big(\vphi\big(\vec f(\vec x)\big)\Big)=\vec f(\vec x).
	\end{equation*}
	Poiché $U\subseteq W$, dove $\vec f$ è iniettiva, dunque se le immagini sono uguali anche le preimmagini devono esserlo, cioè $\vphi\big(\vec f(\vec x)\big)=\vec x$ ossia $\vphi=\vec f^{-1}$ ed è di classe $\cont{1}$ per il teorema di Dini, dato che lo era anche $\vec F$.
	Allora $\vec F$ è un diffeomorfismo in un intorno di $\vec x_0$.
\end{proof}
Anche qui possiamo affermare qualcosa sulle derivate: $\jac\vphi(\vec y)=\jac\vec f^{-1}(\vec y)$ e combinandolo il teorema \ref{t:dini-Rn} di Dini troviamo che
\begin{equation}
	\jac\vec f^{-1}(\vec y)=\bigg[\drp{\vec F}{\vec x}\big(\vec f^{-1}(\vec y)\big)\bigg]^{-1}\cdot I.
\end{equation}
Inoltre dato che $(\vec f^{-1}\circ\vec f)(\vec x)=\vec x$, le loro jacobiane sono tali per cui
\begin{equation*}
	\jac\vec f^{-1}\big(\vec f(\vec x)\big)\jac\vec f(\vec x)=I\qqq \big[\jac\vec f(\vec x)\big]^{-1}=\jac\vec f^{-1}\big(\vec f(\vec x)\big),
\end{equation*}
come dovrebbe giustamente risultare.

\section{Estremi vincolati}
Prendiamo un insieme $K$ compatto di $\R^2$, e una funzione definita da esso a $\R$.
Per il teorema di Weierstrass, se $f\in\cont{}(K)$, deve certamente assumere un massimo ed un minimo in $K$: possiamo distinguerli se si trovano nell'interno o sulla frontiera dell'insieme.
Per i punti in $\mathring{K}$, dove $f$ è differenziabile il teorema di Fermat ci assicura che gli estremanti si trovano laddove $\grad f=\vec 0$.
Sul bordo, però, ciò non è possibile. Trovandoci in $\R^2$, $\partial K$ è una curva di $\R^2$: supponiamo che essa sia regolare, data da una parametrizzazione lungo un intervallo $[a,b]$, cioè sia scrivibile come funzione $\vphi\colon[a,b]\to A\in\R^2$.
Sia dunque $\vphi=\big(x(t),y(t)\big)$: l'insieme $\Phi=\{\vphi(t)\colon t\in[a,b]\}$ è detto \emph{sostegno} della curva $\vphi$.
Restringiamo $f$ all'insieme $\Phi$ e cerchiamo in questo insieme gli estremi: risulta chiaramente che
\begin{equation*}
	\max_{\vec x\in\Phi}f(\vec x)=\max_{t\in[a,b]}f\big(\vphi(t)\big),
\end{equation*}
per una semplice sostituzione.
Sia $F(t)=f\big(\vphi(t)\big)=f\big(x(t),y(t)\big)$: necessariamente deve risultare $F'(t_0)=0$ affinche $t_0$ sia un punto estremante.
Chiamiamo dunque $P\defeq\big(x(t_0),y(t_0)\big)$, e dovremo avere che
\begin{equation}
	\drp{f}{x}(P)x'(t_0)+\drp{f}{y}(P)y'(t_0)=0.
\end{equation}
Possiamo riscrivere questa equazione in termini del gradiente di $f$ come
\begin{equation}
	\grad f(P)\begin{pmatrix} x'(t_0)\\y'(t_0) \end{pmatrix}=\grad f(P)\vphi'(t_0)=0.
\end{equation}
Il vettore $\vphi'(t)$ è sempre tangente a $\vphi(t)$ in ogni punto, quindi quest'equazione significa che il gradiente di $f$ in $P$ è ortogonale alla tangente di $\vphi$ in $t_0$.
Questa è una condizione necessaria affinché $P=\vphi(t_0)$ sia un estremante per $f$ sulla frontiera $\partial K$.
Ciò inoltre significa che il gradiente di $f$ in $P$ è un multiplo del versore normale a $\vphi$ in $t_0$.

Può capitare che la frontiera dell'insieme sia espressa non come parametrizzazione di una curva ma, implicitamente, come luogo di zeri di una certa funzione $\vec F$: essa rappresenta un vincolo a cui ci si restringe per calcolare gli estremi. Abbiamo dunque la seguente definizione.
\begin{definizione} \label{d:estremi-vincolati}
	Siano $A$ un insieme di $\R^n$ e $f$ una funzione da esso a $\R$, un'altra funzione $\vec F\colon A\to\R^m$ (con $m<n$) ed $E$ il luogo degli zeri di quest'ultimo, ossia $E=\{\vec x\in A\colon \vec F(\vec x)=\vec 0\}$.
	Un punto $\vec x_0\in A$ si dice punto di massimo (minimo) \emph{vincolato} di $f$, con $\vec F(\vec x)=\vec 0$ come vincolo, se è un punto di massimo (minimo) per $f|_E$ (cioè per $f$ ristretta all'insieme $E$ come dominio), ossia se esiste un intorno $U(\vec x_0)$ tale che
	\begin{equation*}
		f(\vec x)\leq\ (\geq)\ f(\vec x_0)\ \forall\vec x\in E\cap U(\vec x_0).
	\end{equation*}
\end{definizione}
Abbiamo visto che quando il vincolo è rappresentabile come una curva, in presenza di un punto estremante vincolato il gradiente è proporzionale alla normale a tale curva.
Il seguente teorema mostra una condizione sufficiente ad individuare i punti \emph{stazionari} vincolati (che andranno poi studiati man mano per discernerli).
\begin{teorema}[dei moltiplicatori di Lagrange] \label{t:moltiplicatori-lagrange}
	Sia $A\in\R^2$ aperto, $f\colon A\to\R$ e $F\colon A\to\R$, entrambe di classe $\cont{1}$ in $A$. Sia $E$ il luogo degli zeri di $F$ in $A$. Se $(x_0,y_0)\in A$ è un punto estremante per $f|_E$, e $\grad F(x_0,y_0)\neq(0,0)$, allora esiste un numero $\lambda\in\R$ tale per cui
	\begin{equation} 
		\grad f(x_0,y_0)=\lambda\grad F(x_0,y_0),
		\label{eq:moltiplicatori-lagrange-R2}
	\end{equation}
	ossia $\drp{f}{x}(x_0,y_0)=\lambda\drp{F}{x}(x_0,y_0)$ e $\drp{f}{y}(x_0,y_0)=\lambda\drp{F}{y}(x_0,y_0)$.
\end{teorema}
\begin{proof}
	Possiamo supporre che $\drp{F}{y}(x_0,y_0)\neq 0$ senza ledere la generalità del teorema (sicuramente una tra questa o la derivata rispetto a $x$ deve essere nulla).
	Per il teorema di Dini, l'insieme $E$ in $U(x_0,y_0)$ coincide con il sostegno di una funzione $y=h(x)$ di classe $\cont{1}$, ossia $E\cap U(x_0,y_0)=\{\big(x,h(x)\big)\colon x\in I(x_0)\}$.
	Inoltre, la derivata $h'(x_0)$, che segue sempre dal teorema di Dini, si può scrivere in modo che
	\begin{equation}
		\drp{F}{x}(x_0,y_0)+h'(x_0)\drp{F}{y}(x_0,y_0)=0.
	\end{equation}
	Questo può essere visto come un prodotto tra il gradiente di $F$ in $(x_0,y_0)$ e il vettore $\big(1,h'(x_0)\big)^t$, e dato che è nullo significa che sono ortogonali nel punto $(x_0,y_0)$.
	D'altra parte il punto è vincolato, cioè è un estremante per $f\big(x,h(x)\big)$, che è la stessa cosa che dire di restringere la $f$ a $E\cap U(x_0,y_0)$.
	Derivando la $f$ in questa forma otteniamo
	\begin{equation*}
		f'\big(x_0,h(x_0)\big)=\drp{f}{x}\big(x_0,h(x_0)\big)+\drp{f}{y}\big(x_0,h(x_0)\big)h'(x_0),
	\end{equation*}
	che dev'essere nulla se il punto è un estremo (per il teorema di Fermat: $f$ è differenziabile in $E$).
	Ancora una volta si può interpretare come una relazione di ortogonalità tra il gradiente di $f$ nel punto $(x_0,y_0)=\big(x_0,h(x_0)\big)$ e il vettore $\big(1,h'(x_0)\big)^t$.
	Abbiamo dunque che i gradienti di $f$ e $F$ nel punto sono ortogonali allo stesso vettore: ciò implica che essi sono proporzionali (hanno la stessa direzione), vale a dire $\grad f(x_0,y_0)=\lambda\grad F(x_0,y_0)$ per un certo numero $\lambda\in\R$.
\end{proof}
Il teorema appena dimostrato non si limita ovviamente alla funzioni in due variabili, ma si può generalizzare come di seguito.
\begin{teorema}[dei moltiplicatori di Lagrange] \label{t:moltiplicatori-lagrange}
	Sia $f\colon\Omega\to\R$, con $\Omega\in\R^n$ aperto e $f\in\cont{1}(\Omega)$. Sia inoltre $\vec F\colon\Omega\to\R^m$, con $m<n$, anch'essa di classe $\cont{1}(\Omega)$, e $E=\{\vec x\in\Omega\colon\vec F(\vec x)=\vec 0\}$.
	Se $\vec x_0\in\Omega$ è un estremante di $f|_E$ e il rango della jacobiana $\jac\vec F(\vec x_0)$ è massimo, allora esiste un vettore $\vlambda\in\R^m$ tale che $(\vec x_0,\vlambda)\in\Omega\times\R^m$ è soluzione del sistema
	\begin{equation} 
		\begin{dcases}
			\drp{f}{x_i}(\vec x)=\sum_{j=1}^m\lambda_j\drp{F_j}{x_i}\\
			\vec F(\vec x)=\vec 0
		\end{dcases}
		\label{eq:moltiplicatori-lagrange}
	\end{equation}
	che fornisce $n+m$ equazioni in $n+m$ incognite (le $n$ componenti di $\vec x$ e le $m$ di $\vlambda$).
\end{teorema}
