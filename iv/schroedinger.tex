\chapter{L'equazione di Schrödinger}
\label{ch:schroedinger}

Nella forma più generale, l'equazione di Schrödinger è
\begin{equation}
    -i\hbar\drp{\psi}{t}-\frac{\hbar^2}{2m}\lap\psi+V\psi=0
    \label{eq:schroedinger}
\end{equation}
da risolvere per la funzione $\psi\colon(0,+\infty)\times\R^n\to\C$, di $t$ e $\vec x$, con una funzione fissata $V\colon\R^n\to\R$, di $\vec x$ soltanto, che rappresenta l'energia potenziale; d'ora in poi poniamo $m=1$.
Vogliamo cercare delle soluzioni della forma
\begin{equation}
    \psi(t,\vec x)=e^{-\frac{i}{\hbar}Et}u(\vec x)
    \label{eq:soluzione-stazionaria-schroedinger}
\end{equation}
per un $E\in\R$, ossia delle soluzioni stazionarie della \eqref{eq:schroedinger}: sostituendo questa soluzione abbiamo infatti che $u$ soddisfa
\begin{equation}
    -\frac{\hbar^2}{2}\lap u+Vu-Eu=0
    \label{eq:schroedinger-stazionaria}
\end{equation}
ossia $u$ è un'autofunzione, con autovalore $E$, dell'operatore (i cui insiemi di definizione andranno opportunamente definiti)
\begin{equation}
    -\frac{\hbar^2}{2}\lap+V
    \label{eq:operatore-hamiltoniano}
\end{equation}
detto \emph{hamiltoniano} del sistema.
Studieremo quali condizioni dovrà soddisfare la funzione potenziale $V$ affinch\'e l'equazione differenziale ammetta soluzioni.

\section{Formulazione debole}
\newcommand{\E}{\mathcal{E}}
\renewcommand{\S}{\mathcal{S}}
Introduciamo il funzionale
\begin{equation}
    \E\colon u\mapsto\int_{\R^n}\norm{\grad u}^2\,\dd\mu+\int_{\R^n}Vu^2\,\dd\mu
    \label{eq:funzionale-energia-schroedinger}
\end{equation}
e chiamiamo $\E_0$ l'estremo inferiore che esso assume in $\S=\{u\in\sobH[1]{\R^n}\colon\norm{u}_2=1\}$.
Ha senso restringerci alla sfera unitaria perch\'e l'equazione \eqref{eq:schroedinger-stazionaria} è omogenea quindi se $u$ è soluzione lo è anche $\alpha u$ per qualsiasi $\alpha\in\R$.
Se esiste $\bar{u}\in\S$ tale che $\E(\bar{u})=\E_0$, allora dal teorema di Lagrange \ref{t:lagrange-dimensione-infinita} esiste $\lambda\in\R$ tale che, detto
\begin{equation}
    F\colon u\mapsto\int_{\R^n}u^2\,\dd\mu-1,
\end{equation}
si abbia $\E'(\bar{u})[v]=\lambda F'(\bar{u})[v]$.
Questo significa che
\begin{equation}
    2\int_{\R^n}\bigl(\scalar{\grad\bar{u}}{\grad v}+V\bar{u}v\bigr)\,\dd\mu=2\lambda\int_{\R^n}\bar{u}v\,\dd\mu
\end{equation}
per ogni $v\in\sobH[1]{\R^n}$.
Ponendo $v=\bar{u}$ troviamo dunque
\begin{equation}
    \int_{\R^n}\bigl(\norm{\grad\bar{u}}^2+V\bar{u}^2\bigr)\,\dd\mu=\lambda\int_{\R^n}\bar{u}^2\,\dd\mu
\end{equation}
ma allora $\lambda=\E_0$, e $\bar{u}$ è soluzione debole dell'equazione di Schrödinger.

\begin{teorema}
    Sia $V$ una funzione appartenente a
    \begin{enumerate}
        \item $\leb[\frac{n}2]{\R^n}+\leb[\infty]{\R^n}$, se $n\ge 3$;
        \item $\leb[1+\epsilon]{\R^2}+\leb[\infty]{\R^2}$ per un $\epsilon>0$;
        \item $\leb[1]{\R}+\leb[\infty]{\R}$
    \end{enumerate}
    dove $V\in A+B$ è inteso nel senso che esistono $f\in A$ e $g\in B$ tali che $V=f+g$.
    Allora $\E$ è limitato ed esistono $c,d>0$ tali che
    \begin{equation}
        \int_{\R^n}\norm{\grad u}^2\,\dd\mu\le c\E(u)+d\norm{u}_2^2.
    \end{equation}
\end{teorema}
\begin{proof}
    Ricordiamo dal teorema di Sobolev \ref{t:sobolev} la disuguaglianza\footnote{
        Abbreviamo con $\sobc{p}$ il \emph{coniugato di Sobolev}, ossia l'esponente critico nell'omonimo teorema.
    }
    \begin{equation}
        s_n\biggl(\int_{\R^n}u^{{\sobc 2}}\,\dd\mu\biggr)^\frac{n-2}{n} \le \int_{\R^n}\norm{\grad u}^2\,\dd\mu
    \end{equation}
    se $n\ge 3$, mentre per $n=2$
    \begin{equation}
        s_{2,p}\biggl(\int_{\R^2}u^p\,\dd\mu\biggr)^\frac2{p} \le \int_{\R^2}\bigl(\norm{\grad u}^2+u^2\bigr)\,\dd\mu
    \end{equation}
    e infine per $n=1$
    \begin{equation}
        s_{1,\infty}\norm{u}_\infty^2 \le \int_\R(u^{\prime 2}+u^2)\,\dd\mu
    \end{equation}
    per degli $s_n,s_{2,p},s_{1,\infty}>0$.

    Consideriamo solo il primo caso, per $n\ge 3$: si ha
    \begin{equation}
        \abs[\bigg]{\int_{\R^n}Vu^2\,\dd\mu}\le
        \biggl(\int_{\R^n}\abs{V}^q\,\dd\mu\biggr)^\frac1{q}\biggl(\int_{\R^n}u^{2\frac{{\sobc 2}}2}\,\dd\mu\biggr)^\frac2{{\sobc 2}}
    \end{equation}
    e il valore di $q$ affinch\'e $\frac1{q}+\frac{n-2}2=1$ è $\frac{n}2$.
    Di conseguenza
    \begin{equation}
        \norm{Vu^2}_1\le
        \norm{V}_\frac{n}2\norm{u}_{\sobc 2}^2,
    \end{equation}
    e per il teorema di Sobolev sopra citato
    \begin{equation}
        \norm{Vu^2}_1\le
        \norm{V}_\frac{n}2\frac1{s_n}\int_{\R^n}\norm{\grad u}^2\,\dd\mu.
    \end{equation}
    Siano $v\in\leb[\frac{n}2]{\R^n}$ e $w\in\leb[\infty]{\R^n}$ tali che $V=v+w$ (come dalle ipotesi) e, per $\lambda>0$, definiamo la funzione $h_\lambda\defeq\min\{v+\lambda,0\}$, e mostriamo che esiste $\bar{\lambda}>0$ tale che $\norm{h_{\bar{\lambda}}}_\frac{n}2\le\frac12s_n$.
    Osserviamo infatti che per $\lambda\to+\infty$ la $h_\lambda$ tende puntualmente a zero, mentre $\abs{h_\lambda}\le v$, quindi per il teorema della convergenza dominata \ref{t:convergenza-dominata-Lp} si ha che $\norm{h_\lambda}_\frac{n}2\to 0$ per $\lambda\to+\infty$.
    Possiamo allora effettuare la stima
    \begin{equation}
        \abs[\bigg]{\int_{\R^n}h_\lambda u^2\,\dd\mu}\le
        \norm{h_\lambda}_\frac{n}2\frac1{s_n}\int_{\R^n}\norm{\grad u}^2\,\dd\mu
    \end{equation}
    quindi per $\lambda$ sufficientemente grande possiamo avere $\norm{h_\lambda}_\frac{n}2\le\frac12s_n$: allora
    \begin{equation}
        \abs[\bigg]{\int_{\R^n}h_\lambda u^2\,\dd\mu}\le
        \frac12\int_{\R^n}\norm{\grad u}^2\,\dd\mu.
    \end{equation}
    Il funzionale $\E$, scindendo $V$ nei due addendi, è dunque tale per cui
    \begin{equation}
        \begin{split}
            \E(u)&=
            \int_{\R^n}\bigl[\norm{\grad u}^2+(v+\lambda)u^2-\lambda u^2+wu^2\bigr]\,\dd\mu\ge\\ &\ge
            \int_{\R^n}\norm{\grad u}^2\,\dd\mu+\int_{\R^n}h_\lambda u^2\,\dd\mu-\lambda\int_{\R^n}u^2\,\dd\mu\int_{\R^n}wu^2\,\dd\mu\ge\\ &\ge
            \int_{\R^n}\Bigl(\norm{\grad u}^2-\frac12\norm{\grad u}^2-\lambda u^2-\norm{w}_\infty u^2\Bigr)\,\dd\mu
        \end{split}
    \end{equation}
    e siccome $\norm{u}_2=1$ otteniamo
    \begin{equation}
        \E(u)\ge
        \frac12\int_{\R^n}\norm{\grad u}^2\,\dd\mu-\lambda-\norm{w}_\infty\ge
        -\lambda-\norm{w}_\infty
    \end{equation}
    che è dunque un limite inferiore di $\E$.
    Allo stesso tempo
    \begin{equation}
        \int_{\R^n}\norm{\grad u}^2\,\dd\mu\le
        c\E(u)+d
    \end{equation}
    dove $c=2$ e $d=\lambda+\norm{w}_\infty$.
\end{proof}
\begin{osservazione} \label{o:potenziali-ammissibili}
    Queste tipologie di potenziali non esauriscono quelli per i quali l'equazione di Schrödinger ammette una soluzione (come l'oscillatore armonico, che chiaramente non appartiene ad alcun $\leb[p]{\R^n}$ per qualsiasi $p\ge 1$), ma comprendono i casi ``più realistici'' che non divergono all'infinito e ammettono eventualmente delle singolarità che vanno a $-\infty$, come il potenziale coulombiano.
    Per questi tipi di potenziale un'energia positiva è tipicamente associata a uno stato non legato.
\end{osservazione}
