\chapter{L'equazione delle onde}
\label{ch:onde}

L'equazione delle onde, o anche di D'Alembert, è
\begin{equation}
    \ddrp{u}{t}-\lap u=f
    \label{eq:onde}
\end{equation}
per $u=u(t,\vec x)$ con $t\in[0,+\infty)$ e $\vec x\in\Omega\subset\R^n$.
Più in breve, definendo l'\emph{operatore dalembertiano} $\dal=\ddrp{}{t}-\lap$ possiamo riscriverla come
\begin{equation}
    \dal u=f.
    \label{eq:onde-dalembertiano}
\end{equation}
Per semplicità, trattiamo il caso $n=1$, $\Omega=\R$ e poniamo il problema di Cauchy
\begin{equation}
    \begin{cases}
        \ddrp{u}{t}-\ddrp{u}{x}=0 & \text{in }(0,+\infty)\times\R\\
        u=g                       & \text{in }\{0\}\times\R\\
        \drp{u}{t}=h              & \text{in }\{0\}\times\R
    \end{cases}
    \label{eq:problema-cauchy-onde}
\end{equation}
e cerchiamo una soluzione in termini di $g$ e $h$.
Mediante un'opportuna fattorizzazione possiamo ricondurre l'equazione omogenea a una coppia di equazioni del trasporto:
\begin{equation}
    \dal u=\biggl(\drp{}{t}+\drp{}{x}\biggr)\biggl(\drp{u}{t}-\drp{u}{x}\biggr)=0
\end{equation}
e detto $v=\drp{u}{t}-\drp{u}{x}$ abbiamo
\begin{equation}
    \drp{v}{t}+\drp{v}{x}=0.
\end{equation}
Riconducendoci all'esempio \ref{es:equazione-trasporto}, sia $a(x-t)\defeq v(t,x)$; risulta $\drp{v}{x}(t,x)=a'(x-t)$ e $\drp{v}{t}(t,x)=-a'(t,x)$.
D'altro canto l'equazione
\begin{equation}
    \drp{u}{t}(t,x)-\drp{u}{x}(t,x)=a(x-t)
\end{equation}
è ancora un'equazione del trasporto (non omogenea), per cui
\begin{equation}
    u(t,x)=
    \int_0^ta\bigl(x+(t-s)-s)\,\dd s+b(x+t)=
    \frac12\int_{x-t}^{x+t}a(y)\,\dd y+b(x+t)
\end{equation}
dove $a$ è determinata dalla non omogeneità della soluzione mentre $b$ è determinata dai dati iniziali: si ha $b(x)=u(0,x)=g(x)$ e $a(x)=v(0,x)=\drp{u}{t}(0,x)-\drp{u}{x}(0,x)=h(x)-g'(x)$ quindi
\begin{equation}
    \begin{split}
        u(t,x)&=
        \frac12\int_{x-t}^{x+t}\bigl[h(y)-g'(y)\bigr]\,\dd y+g(x+t)=\\ &=
        \frac12\int_{x-t}^{x+t}h(y)\,\dd y+\frac12\bigl[g(x-t)-g(x+t)\bigr]+g(x+t)=\\ &=
        \frac12\int_{x-t}^{x+t}h(y)\,\dd y+\frac12\bigl[g(x-t)+g(x+t)\bigr]
    \end{split}
    \label{eq:rappresentazione-soluzione-onde}
\end{equation}
che è la formula di rappresentazione generale.
Essa vale per $t>0$, ma si dimostra che se $g\in\cont[2]{\R}$ e $h\in\cont[1]{\R}$ allora $u$ soddisfa l'equazione delle onde su tutto $\R$.
\begin{teorema}
    Se $g\in\cclass[2]$ e $h\in\cclass[1]$ allora la soluzione $u$ del problema di Cauchy \eqref{eq:problema-cauchy-onde} è in $\cont[2]{(0,+\infty)\times\R}$ e vale
    \begin{equation}
        g(x_0)=\lim_{(t,x)\to(0,x_0)}u(t,x)
        \qtext{e}
        h(x_0)=\lim_{(t,x)\to(0,x_0)}\drp{u}{t}(t,x).
    \end{equation}
\end{teorema}
