\chapter{Numeri complessi}
I numeri complessi nascono come soluzione all'equazione $x^2=-1$, e in generale a quei problemi che richiedono la radice (di indice pari) di un numero negativo: si introduce così l'\emph{unità immaginaria}, indicata con $i$, tale per cui
\[
i^2=-1.
\]
Con questa nuova unità si costruiscono i numeri \emph{immaginari puri}, ovvero i numeri della forma $i\alpha$, con $\alpha\in\R$: a questo punto, l'equazione (ad esempio) $x^2=-\alpha$, se $\alpha$ è positivo, ha sempre delle soluzioni, in questo caso il numero $i\sqrt{\alpha}$ e il suo opposto.
Si chiama inoltre \emph{numero complesso} una qualunque espressione del tipo $z=a+ib$, con $a,b\in\R$. Tale scrittura è detta \emph{forma algebrica} di $z$, e $a$ si chiama parte reale di $z$, mentre $b$ è la parte immaginaria, e si possono indicare come
\[
a=\Re z,\qquad b=\Im z.
\]
La coppia ordinata $(a,b)$ può essere inoltre associata ad un vettore di $\R^2$, e in effetti i numeri complessi sono in corrispondenza biunivoca con $\{(a,b)\colon a,b\in\R\}$. I numeri complessi quindi si rappresentano su un piano, chiamato appunto \emph{piano complesso}, dove l'asse delle ascisse è l'asse dei numeri reali, e l'asse delle ordinate è l'asse dei numeri immaginari; è chiaro quindi come i numeri reali siano compresi nei numeri complessi: sono semplicemente dei numeri complessi la cui parte immaginaria è nulla.

\section{Operazioni sui numeri complessi}
Dati due numeri $z=a+ib$ e $w=c+id$, si definiscono:
\begin{itemize}
\item l'operazione di somma, come $z+w=(a+c)+i(b+d)$;
\item l'operazione di prodotto, come $zw=(ac-bd)+i(bc+ad)$ (è un semplice prodotto tra due polinomi);
\end{itemize}
\begin{teorema}
L'insieme dei numeri complessi, indicato con $\C$, dotato delle operazioni appena definite, è un campo, indicato anche con $(\C,+,\cdot)$.
\end{teorema}
Ovviamente tutte le operazioni appena descritte valgono anche per i numeri reali e si riducono a quelle già note, poiché $\R\subset\C$.

L'elemento neutro per la somma è $(0,0)$, mentre l'opposto di $z=a+ib$ è $-z=-a-ib$, cioè il simmetrico rispetto all'origine.
L'elemento neutro per il prodotto è $(1,0)$, e il reciproco di $z=a+ib$ (non nullo) è il numero complesso $\frac1{z}=x+iy$ per cui il prodotto $(x+iy)(a+ib)$ sia 1, quindi
\[
\begin{cases}
	ax-by=1\\
	bx+ay=0
\end{cases}
\longrightarrow
\begin{dcases}
	x=\frac{a}{a^2+b^2}\\
	y=-\frac{b}{a^2+b^2}
\end{dcases}
\]

Dato il numero $z=a+ib$, si definiscono inoltre il suo \emph{coniugato}, cioè il numero simmetrico rispetto all'asse reale, $\conj{z}=a-ib$, e il suo \emph{modulo} $\abs{z}=\sqrt{a^2+b^2}$, che è la  distanza del punto $(a,b)$ dall'origine.
Si hanno le seguenti operazioni con i coniugati e i moduli:
\begin{itemize}
\item $\conj{z+w}=\conj{z}+\conj{w}$;
\item $\conj{zw}=\conj{z}\cdot\conj{w}$;
\item $z+\conj{z}=2\,\Re z$;
\item $z-\conj{z}= 2i\,\Im z$;
\item $z\conj{z}=\abs{z}^2$.
\item $\abs{zw}=\abs{z}\cdot\abs{w}$;
\item $\abs{\conj{z}}=\abs{z}$;
\item $\abs{z+w}\leq\abs{z}+\abs{w}$.
\end{itemize}
\begin{proof}
L'ultima disuguaglianza, che è la disuguaglianza triangolare, si dimostra nel modo seguente:
\[
\abs{z+w}^2=(z+w)\cdot(\conj{z+w})=(z+w)(\conj{z}+\conj{w})=z\conj{z}+w\conj{w}+z\conj{w}+\conj{z}w.
\]
Gli ultimi due termini sono l'uno il coniugato dell'altro, quindi
\[
z\conj{z}+w\conj{w}+z\conj{w}+\conj{z}w=\abs{z}^2+\abs{w}^2+2\,\Re{z\conj{w}}.
\]
La parte reale di un numero è sempre minore o uguale al suo modulo, quindi l'uguaglianza rimane applicando i valori assoluti: si possono vedere come il cateto e l'ipotenusa di un triangolo rettangolo, rispettivamente. Quindi $\abs{\Re z}\leq\abs{z}$, e analogamente $\abs{\Im z}\leq\abs{z}$. Quindi si ha
\[
\begin{split}
\abs{z}^2+\abs{w}^2+2\,\Re{z\conj{w}}&\leq\abs{z}^2+\abs{w}^2+2\abs{z\conj{w}}=\\
&=\abs{z}^2+\abs{w}^2+2\abs{z}\cdot\abs{w}=\\
&=(\abs{z}+\abs{w})^2.
\end{split}
\]
Applicando le radici, si ottiene quindi la disuguaglianza $\abs{z+w}\leq\abs{z}+\abs{w}$.
\end{proof}
Con queste definizioni diventa più ``comodo'' individuare il reciproco di un numero complesso, che si scrive anche come
\[
\frac1{z}=\frac{\conj{z}}{z\conj{z}}=\frac{\conj{z}}{\abs{z}^2},
\]
che è un prodotto tra un numero reale e un numero complesso.
Infatti
\[
\frac{\conj{z}}{\abs{z}^2}=\frac{a-ib}{a^2+b^2}=\frac{a}{a^2+b^2}-i\frac{b}{a^2+b^2}.
\]
Il rapporto tra due numeri complessi allora si scrive come il prodotto di uno per il reciproco dell'altro, quindi
\[
\frac{z}{w}=z\cdot\frac{\conj{w}}{\abs{w}^2}.
\]

\section{Forma trigonometrica ed esponenziale}
Un altro modo per rappresentare i numeri complessi si ottiene guardandoli in coordinate polari anziché cartesiane: quindi al posto delle componenti sui due assi, cioè la parte reale e la parte immaginaria, si individua il numero con la sua distanza dall'origine e l'angolo che forma con l'asse reale (definito a meno di multipli di $2\pi$), chiamati rispettivamente \emph{modulo}, $r$, e \emph{argomento}, $\theta$:
\[
z=r(\cos\theta+i\sin\theta),
\]
che vale se $z\neq 0$, poiché in tal caso non si può determinare l'argomento.
Si può passare da una definizione all'altra tramite le formule
\[
\begin{dcases}
	a=r\cos\theta\\
	b=r\sin\theta
\end{dcases}
\quad\text{e}\quad
\begin{dcases}
	r=\sqrt{a^2+b^2}\\
	\cos\theta=a/r\\
	\sin\theta=b/r
\end{dcases}
\]
\begin{figure}
	\tikzsetnextfilename{forma-trigonometrica-complesso}
	\centering
	\begin{tikzpicture}
		\draw [->] (0,0) -- (0,4) node[anchor=south]{$\mathrm{Im}$};
		\draw [->] (0,0) -- (5,0) node[anchor=west]{$\mathrm{Re}$};
		\draw (0,0) node[anchor=north east]{$O$};
		\draw (0,0) to node[auto]{$r$} (4,3) node[anchor=south west]{$z=re^{i\theta}$};
		\draw [dotted] (4,0) -- (4,3);
		\draw [dotted] (0,3) -- (4,3);
		\draw (4,0) node[anchor=north]{$a$};
		\draw (0,3) node[anchor=east]{$b$};
		\draw (1.5,0) arc (0:36.87:1.5) node[anchor=west] at (18.435:1.5){$\theta$};
	\end{tikzpicture}
	\caption{Forma trigonometrica ed algebrica di un numero complesso.}
	\label{fig:forma-trig}
\end{figure}


Questo modo di scrittura dei numeri complessi facilita le operazioni di prodotto, in quanto semplicemente si moltiplicano i moduli e si sommano gli argomenti: dati $z=r(\cos\theta+i\sin\theta)$ e $w=\rho(\cos\phi+i\sin\phi)$, si ha
\[
zw=r\rho\big(\!\cos(\theta+\phi)+i\sin(\theta+\phi)\big).
\]
\begin{proof}
Utilizzando le formule del seno e coseno della somma di angoli, risulta
\[
\begin{split}
zw&=r\rho(\cos\theta\cos\phi+i^2\sin\theta\sin\phi+i\sin\theta\cos\phi+i\cos\theta\sin\phi)=\\
&=r\rho\big(\!\cos(\theta+\phi)+i\sin(\theta+\phi)\big)\qedhere.
\end{split}
\]
\end{proof}
Si nota come moltiplicare per un numero complesso equivale, oltre ovviamente a moltiplicare i due moduli, a ruotare uno di un angolo equivalente all'argomento dell'altro.
Inoltre il reciproco di $z\neq 0$, in questa forma, è
\[
\frac1{z}=\frac1{r}\big(\!\cos(-\theta)+i\sin(-\theta)\big)=\frac1{r}(\cos\theta-i\sin\theta).
\]
Per il rapporto, invece, si sottraggono gli argomenti anziché sommarli.

Un'ulteriore modo più ``compatto'' di rappresentare i numeri complessi è in forma esponenziale, ossia nella scrittura
\[
z=re^{i\theta},
\]
dove $r$ e $\theta$ sono esattamente lo stesso modulo e argomento della scrittura in forma trigonometrica. La moltipilcazione è ancora più veloce:
\[
zw=r\rho e^{i(\theta+\phi)},
\]
e così via.
\section{Potenze e radici}
Si definiscono le potenze di numeri complessi ad esponente intero relativo, cioè i numeri $z^n$ con $n\in\Z$.
Sia $z=r(\cos\theta+i\sin\theta)$. La sua potenza $n$-esima è il numero moltiplicato per se stesso $n$ volte. Seguendo la formula per i prodotti in forma trigonometrica,
\[
z^n=\underbracket[.4pt]{r\times r\times\dots\times r}_\text{$n$ volte}\big(\!\cos(\underbracket[.4pt]{\theta+\theta+\dots+\theta}_\text{$n$ volte})+i\sin(\underbracket[.4pt]{\theta+\theta+\dots+\theta}_\text{$n$ volte})\big)
\]
Il suo modulo sarà quindi $r$ elevato all'esponente $n$, e l'argomento sarà $\theta$ aggiunto a se stesso $n$ volte, quindi $n\theta$. Quindi si ha la formula detta \emph{di De Moivre}:
\begin{equation}
\label{eq:demoivre}
z^n=r^n\big(\!\cos(n\theta)+i\sin(n\theta)\big).
\end{equation}
Si definisce inoltre $z^0=1$, e se l'esponente è negativo si pone $z^n=\big(\frac1{z}\big)^{-n}$ e si torna al caso precedente.
\begin{definizione}
\label{d:radici-c}
Sia $w\neq 0$ un numero complesso, e $n\geq 2$. Il numero $z\in\C$ è \emph{radice} $n$-esima di $w$ se $z^n=w$.
\end{definizione}
Nel campo complesso, le radici di un equazione di grado $n$ sono sempre $n$, tutte distinte.
\begin{teorema}
\label{t:radici-c}
Sia $w\neq 0$, con $w=r(\cos\theta+i\sin\theta)$ e $n\geq 2$ ($n\in\N$). Esistono sempre $n$ numeri complessi $z_k$ che sono radici $n$-esime di $w$, e sono tutti e solo numeri complessi della forma
\begin{equation}
\label{eq:radici-c}
z_k=\sqrt[n]{r}\left(\cos\frac{\theta+2k\pi}{n}+i\sin\frac{\theta+2k\pi}{n}\right),
\end{equation}
con $k=0,1,\dots,n-1$.
\end{teorema}
Partendo dalla prima radice, di argomento $\theta/n$, le altre sono ruotate di $2k\pi/n$ per ogni $k=0,1,\dots,n-1$.
Le $n$ radici allora dividono la circonferenza di raggio $\sqrt[n]{r}$, centrata nell'origine, in $n$ angoli o archi uguali: i punti $(r,\theta)$ che sono radici del numero complesso di partenza formano un poligono equilatero inscritto in tale circonferenza.
\begin{figure}
\tikzsetnextfilename{radici-complesse}
\centering
\begin{tikzpicture}
\draw (0,0) circle (2cm);
\draw (0,0) node[above left]{$O$};
\draw [->] (-2.5cm,0cm) -- (2.5cm,0cm) node[right]{$\mathrm{Re}$};
\draw [->] (0cm,-2.5cm) -- (0cm,2.5cm) node[above]{$\mathrm{Im}$};
\draw [dashed] (60:2cm) -- (180:2cm) -- (300:2cm) -- (60:2cm);
\draw (60:2cm) node[small dot,pin={[pin distance=.3cm]60:$e^{i\frac{\pi}3}$}]{};
\draw (180:2cm) node[small dot,pin={[pin distance=.3cm]135:$-1$}]{};
\draw (300:2cm) node[small dot,pin={[pin distance=.3cm]300:$e^{-i\frac{\pi}3}$}]{};
\end{tikzpicture}
\caption{Le tre radici cubiche di -1, che formano un triangolo equilatero inscritto nella circonferenza di raggio $r=1$.}
\end{figure}

\begin{proof}
Sia $z=\rho(\cos\phi+i\sin\phi)$ una radice $n$-esima del numero $w=r(\cos\theta+i\sin\theta)$, dove ovviamente $w,z\in\C$. Allora deve essere $z^n=w$, quindi
\[
\rho^n\big(\!\cos(n\phi)+i\sin(n\phi)\big)=r(\cos\theta+i\sin\theta).
\]
Perché questi due numeri siano uguali, deve allora essere che $\rho^n=r$ e $n\phi=\theta+2h\pi$, per qualche $h\in\Z$.
La prima è un'uguaglianza tra due numeri reali, quindi ha un'\emph{unica} soluzione positiva che è $\rho=\sqrt[n]{r}$. Inoltre $\phi=(\theta+2h\pi)/n$ per qualche $h\in\Z$.
Però se si prendono $h$ e $h+n$, i numeri $z$ corrispondenti hanno lo stesso argomento, che è definito a meno di multipli di $2\pi$:
\[
\frac{\theta+2(h+n)\pi}{n}=\frac{\theta+2h\pi}{n}+2\pi=\frac{\theta+2h\pi}{n}.
\]
Quindi, in realtà, le radici si ripetono ciclicamente tutte uguali dopo ogni multiplo di $n$, allora esistono solo $n$ radici \emph{uniche}.
\end{proof}
\begin{teorema}[Teorema fondamentale dell'algebra]
Il campo $\C$ è algebricamente chiuso. Qualunque polinomio a coefficienti complessi ha sempre un numero di soluzioni (complesse) pari al grado del polinomio.
\end{teorema}
Il fatto che esistano sempre $n$ radici di un polinomio di grado $n$ nel campo complesso, e che $\R\subset\C$, significa anche che ogni polinomio di grado $n$ a coefficienti reali ha sempre $n$ soluzioni in $\C$, cioè che $\C$ è la \emph{chiusura algebrica} di $\R$.
\section{Ordinamento}
Se come già visto si possono ordinare solo coppie di numeri reali (o di sottoinsiemi di $\R$), in effetti non è possibile stabilire un'ordinamento tra i numeri complessi. Qualunque tipo di ordine si voglia stabilire non è alla fine compatibile con le operazioni precedentemente definite.
\begin{proof}
Si consideri il numero $i$: poiché non è nullo, dovrebbe essere o positivo o negativo. Se $i>0$, allora il suo quadrato è positivo. Sia quindi $i\cdot i=-1$ positivo. Moltiplicando due numeri positivi, il prodotto deve essere positivo: allora sia $i\cdot (-1)=-i$ positivo. Ora, la somma di due numeri positivi deve essere a sua volta positiva, ma se si somma $i$ e $-i$ si ottiene 0, che non è positivo. Questo assurdo porta alla conclusione che $i$ non può essere positivo.
Sia allora $i$ negativo. Il suo quadrato deve essere positivo, e come prima allora sia $-1$ positivo. Moltiplicando un numero negativo e un numero positivo, rispettivamente $i$ e $-1$, si ottiene $-i$ che deve allora essere negativo. La somma di due numeri negativi deve essere quindi negativa, ma sommando $i$ e $-i$ si ottiene 0 che non è negativo. Per questo altro assurdo, $i$ non può essere nemmeno negativo.
Allora $i$ non è né negativo, né positivo, né nullo, quindi $\C$ non si può ordinare compatibilmente con le sue operazioni.
\end{proof}
